\part*{ЗАКЛЮЧЕНИЕ}
\addcontentsline{toc}{part}{\textbf{ЗАКЛЮЧЕНИЕ}} 

В рамках данной работы было проведено изучение системы обнаружения аудиодипфейков, анализ предметной области, рассмотрение признаков аудио для изучения и обучения, а также проведена классификация и обзор методов обнаружения аудиодипфейков.

В итоге можно описать структуру работы системы обнаружения синтетического звука следующим образом:

\begin{enumerate}
    \item На вход поступает аудиозапись.
    \item Модель извлечения признаков предварительно обрабатывает запись.
    \item В некоторых методах модели используют признаки для обучения, а в некоторых других просто принимают звуковую речь.
    \item Модель классификации использует признаки или саму звуковую речь для обучения и распознавания.
\end{enumerate}

Значительную роль в эффективности работы модели играют признаки, передаваемые в модель для классификации, а также сам процесс работы модели классификации.

Следует отметить, что огромное значение для работоспособности и результатов модели по классификации имеют корпусы данных, используемые для обучения и извлечения признаков, которое обнаруживается в процессе обучения модели.
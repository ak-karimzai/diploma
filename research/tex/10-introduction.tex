\part*{ВВЕДЕНИЕ}

Аудио-дипфейки представляют собой категорию звуковых файлов, созданных при помощи глубоких нейронных сетей, способных анализировать и воспроизводить звуковые характеристики настолько реалистично, что созданный контент может звучать естественно и непринужденно. Чаще всего эти технологии применяются для имитации голосов людей, но, кроме того, могут вызывать веселье и забаву. Тем не менее, с увеличением популярности аудио-дипфейков возникают вопросы относительно их злоупотребления с целью распространения дезинформации.

Хотя синтетический или фейковый контент существует уже много лет, внимание к контенту, созданному с использованием нейронных сетей, также известного как дипфейк, стало значительным лишь в последние несколько лет. В то время как синтезированные фотографии и видео, порожденные нейронным сетям, привлекли большое внимание, синтетические человеческие голоса тоже достигли выдающегося качества и эффективности. Но, несмотря на их улучшенную реалистичность и доступность, синтетические голоса также несут в себе существенные риски.

В рамках данной научной-исследовательской работы рассмотрим следующие цели и задачи:

\begin{enumerate}
    \item Синтезирование аудио: Понятие и Типы;
    \item Характеристики и особенности аудиоматериала для изучения;
    \item Понятие и схема работы системы обнаружения синтетического звука;
    \item Классификация и обзор известных методов обнаружения синтезированного звука.
\end{enumerate}


\addcontentsline{toc}{part}{\textbf{ВВЕДЕНИЕ}}
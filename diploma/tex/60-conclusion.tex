\part*{ЗАКЛЮЧЕНИЕ}
\addcontentsline{toc}{part}{\textbf{ЗАКЛЮЧЕНИЕ}} 

В рамках данной выпускной квалификационной работы было разработано программное обеспечение и метод для устранения размытия изображений. Результаты работы можно описать следующим образом:

\begin{itemize}[left=0.49cm]
    \item Было изучено и проведено сравнение различных видов размытия изображений;
    \item Были изучены и освоены основные принципы работы нейронных сетей;
    \item Были рассмотрены и выбраны существующие методы нейронных сетей для работы с изображениями;
    \item Были рассмотрены и выбраны архитектуры сетей, решающих задачу устранения размытия;
    \item Были сформулированы требования к разработке программного обеспечения, методу и выбору корпусов данных;
    \item На основе сформулированных требований было разработано программное обеспечение, метод и выбраны соответствующие корпусы данных;
    \item Было проведено обучение и валидация сети на выбранных корпусах данных;
    \item Для демонстрации обобщаемости было проведено тестирование сети на различных видах изображений из разных корпусов данных, которые не использовались в обучении;
    \item Был проведен сравнительный анализ метода по объективным метрикам, и были составлены таблицы с результатами.
\end{itemize}

% \newenvironment{itemize*}%
%   {\begin{itemize}%
%     % \setlength{\itemsep}{0pt}%
%     \setlength{\parskip}{0pt}}%

%     \item Было изучено и проведено сравнение различных видов размытия изображений;
%     \item Были изучены и освоены основные принципы работы нейронных сетей;
%     \item Были рассмотрены и выбраны существующие методы нейронных сетей для работы с изображениями;
%     \item Были рассмотрены и выбраны архитектуры сетей, решающих задачу устранения размытия;
%     \item Были сформулированы требования к разработке программного обеспечения, методу и выбору корпусов данных;
%     \item На основе сформулированных требований было разработано программное обеспечение, метод и выбраны соответствующие корпусы данных;
%     \item Было проведено обучение и валидация сети на выбранных корпусах данных;
%     \item Для демонстрации обобщаемости было проведено тестирование сети на различных видах изображений из разных корпусов данных, которые не использовались в обучении;
%     \item Был проведен сравнительный анализ метода по объективным метрикам, и были составлены таблицы с результатами.
%   {\end{itemize}}

В итоге была достигнута поставленная цель работы, и было разработано программное обеспечение и метод, решающие поставленную задачу.

\part*{ОПРЕДЕЛЕНИЯ, ОБОЗНАЧЕНИЯ И\\СОКРАЩЕНИЯ}
\addcontentsline{toc}{part}{\textbf{ОПРЕДЕЛЕНИЯ, ОБОЗНАЧЕНИЯ И СОКРАЩЕНИЯ}}
В настоящей расчетно-пояснительной записке применяют следующие термины с соответствующими определениями.

\begin{enumerate}[left=0.49cm]
	\item \textbf{CNN} (\textit{англ. convolutional neural network}) -- сверточная нейронная сеть.
	\item \textbf{ResNet} (\textit{англ. residual neural network}) -- остаточная нейронная сеть.
	\item \textbf{CapsNet} (\textit{англ. capsule neural network}) -- капсульная нейронная сеть.
	\item \textbf{PSF} (\textit{англ. point spread function}) -- функция рассеяния точки.
	\item \textbf{DMPHN} (\textit{англ. deep stacked multi-patch hierarchical network}) -- глубокая многоуровневая иерархическая сеть с множеством патчей.
	\item \textbf{MPRNet} (\textit{англ. multi-stage progressive hierarchical network}) -- многоуровневая прогрессивная иерархическая сеть.
	\item \textbf{MSE} (\textit{англ. mean squared error}) -- среднеквадратичная ошибка.
	\item \textbf{PSNR} (\textit{англ. peak signal-to-noise ratio}) -- пиковое отношение сигнал/шум.
	\item \textbf{SSIM} (\textit{англ. structural similarity measure}) -- структурная мера сходства.
	\item \textbf{ORS} (\textit{англ. original resolution subnetwork}) -- подсеть с оригинальным разрешением.
	\item \textbf{ORB} (\textit{англ. original resolution block}) -- блок с оригинальным разрешением.
	\item \textbf{CAB} (\textit{англ. channel attention blocks}) -- блоки внимания к каналам.
	\item \textbf{CSFF} (\textit{англ. cross-stage feature fusion}) -- кросс-стейдж слияние признаков.
	\item \textbf{CLI} (\textit{англ. command line interface}) -- интерфейс командной строки.
\end{enumerate}

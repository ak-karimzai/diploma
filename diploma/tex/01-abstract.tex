\part*{РЕФЕРАТ}
%\thispagestyle{empty}
\addcontentsline{toc}{part}{\textbf{РЕФЕРАТ}}

Расчетно--пояснительная записка \pageref{LastPage} с., \totalfigures\ рис., \totaltables\ табл., \thetotalbibentries\ ист, 1 прил.

В работе рассматривается понятие размытия изображения, существующие типы и методы его устранения. Также проводится постановка задачи, решающей проблему размытия. Далее описывается метод и его реализация для устранения размытия изображения, а также проводится исследование разработанного метода и рассматривается оценка его обобщаемости.

\textbf{Ключевые слова:} Размытие изображения, устранение размытия изображения, нейронные сети, сверточные нейронные сети, многоэтапная архитектура нейронных сетей, билинейная интерполяция.
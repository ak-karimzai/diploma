\part*{ВВЕДЕНИЕ}
\addcontentsline{toc}{part}{\textbf{ВВЕДЕНИЕ}}

Устранение размытия изображения — это задача, в которой главной целью является устранение элементов, вызывающих размытость, и улучшение качества изображения для более ясной визуализации текстур и объектов \cite{shen2018deep}, с другой точки зрения в задачи рассмотривается улучшение текстуру и качество изображений для дальнейшего использования в задачах машинного зрения, таких как обнаружение объектов и сегментация изображений, шум и атмосферные помехи, движение объектов, дрожание камеры и оборудование для расфокусировки являются распространенными источниками ухудшения качества изображения \cite{NIPS2009_3dd48ab3}.

Устранение размытия изображения остается значимой задачей в сфере обработки изображений и компьютерного зрения. С развитием фотографии и видеосъемки возникает постоянная потребность в повышении качества изображений, особенно в условиях недостаточного освещения, движущихся объектов и других факторов, способных вызвать размытие. Методы улучшения четкости изображений постоянно совершенствуются, включая разработки в области машинного и глубокого обучения, что позволяет достигать более точных и эффективных результатов \cite{Lian2023}, Поэтому задача привлекает внимание исследователей и разработчиков программного обеспечения. Применение улучшенных изображений находит свое применение в различных областях, включая медицину, робототехнику, автомобильную промышленность и другие. Также стоит отметить, что улучшение качества изображений играет важную роль в раскрытии и расследовании преступлений. Более четкое изображение может существенно улучшить возможности идентификации объектов и лиц, что помогает в расследовании криминальных деяний \cite{sun2010}.

Существует ряд алгоритмов, решающих данную задачу, но в последние десятилетия все разработанные методы, обычно используемые в области обработки изображений и компьютерного зрения, применяют в своих архитектурах нейронные сети и показывают наилучшие результаты по сравнению с классическими алгоритмами \cite{zhang2022deep}. 

Целью данной выпускной квалификационной работы является:

\begin{itemize}
    \item Размытие изображения: Понятие и виды;
    \item Понятие и схема работы алгоритмов, решающих данную задачу;
    \item Провести анализ существующих методов и алгоритмов, и выбрать подходящий для решения данной задачи;
    \item Подобрать датасеты, содержащие изображения нескольких людей на фоне природы для исследования;
    \item Провести анализ и оценку качества полученных результатов. 
\end{itemize}